\documentclass[11pt]{article}

\usepackage{graphicx}
\usepackage{float}
\usepackage{amsmath}
\usepackage{listings}
\usepackage[a4paper, total={7in, 10in}]{geometry}
\usepackage[utf8x]{inputenc}
\usepackage[russian]{babel}
\title{Изучение связи свойств темного гало и крупномасштабной структуры Вселенной}
\author{Андрей\,Знобищев}
\date{19 декабря 2016}
\graphicspath{ {D:/Pictures/} }

\begin{document}

\maketitle

\begin{abstract}
	Данная работа является отчетом о научно-исследовательской работе за 7 семестр 2016/2017 учебного года в рамках учебной, преддипломной практики студента 326 группы Факультета Общей и Прикладной Физики Москвоского Физико-Технического Института (Государственного Университета). \\ 
	Базовая организация, кафедра: Физический Институт Академии Наук им. П.Н. Лебедева, кафедра проблем физики и астрофизики.
\\
	Научный руководитель: Сергей Владимирович Пилипенко. 
\end{abstract}

\section{Вступленние}
Современные наблюдения большого числа галактик в областях размера нескольких мегапарсек показывают, что распределение галактик неоднородно. (Рис.1 и Рис.2). \\
Эта неоднородность является результатом роста малых флуктуаций плотности, которые возникли из-за гравитационной нестабильности в ранней Вселенной. Существует теория, описывающая данный процесс в общем. Также есть возможность численно моделировать рост флуктуаций и образование гравитационно связанных структур. Однако, как мы можем описывать наблюдательные данные количественно и сравнивать их с теорией и численным моделированием? \\
Одна из базовых величин - это функция масс n(M) космологических структур, определяемая как\\
\begin{equation}
	dN = n(M)dM,
\end{equation}
где dN - число объектов на единицу объема с массой в промежутке от M до M+dM. 
 
\begin{figure}[h]
\centering
\includegraphics[width=0.4\textwidth]{SDSS.jpg}
\caption{Карта Вселенной, полученная из наблюдений SDSS. Расстояние до галактик выражается в терминах красного смещения}
\end{figure}
\begin{figure}[H]
\centering
\includegraphics[width=0.4\textwidth]{slice.png}
\caption{Один из "срезов" симуляции MDR1 при красном смещении z = 0.53 }  
\end{figure}

 С наблюдательной точки зрения, процесс подсчета функции масс очевиден. Необходимо выбрать объекты заданной массы в указанной части пространства. Однако, удобно было бы иметь аналитическую формулу для выражения функции масс. И теоретическая работа получения данной функции была выполнена в 1974 году Прессом и Шехтером[2]. Следующее ниже известно как "Формализм Пресса-Шехтера". \\

Обозначим $\delta$(x,t) флуктуацию плотности (далее временная зависимость опускается).  Другой параметр R: информация о поведении $\delta$(x) на масштабах меньше R не учитывается. Для этого используется "функция-окно"  W(x - x'):
\begin{equation}
	\delta(x,R) = \int d^3x \delta(x')W_{R}(x-x').
\end{equation}
Возможен разичный выбор функции-окна. В данной работе используется функция \textit{"top hat"}:
\begin{equation}
	\begin{split}
	W (x - x') = 1,\ |x-x'|<R,\\ = 0,\ |x-x'| \geq  R
	\end{split}
\end{equation}  
Сфера радиуса R содержит массу
\begin{equation}
M = \frac{4\pi}{3}\rho_0 R^3,
\end{equation}
Так как существует прямая связь R и M, можно обозначать $\delta(x,R)=\delta(x,M) \equiv \delta_M.$
Ключевое предположение теории Пресса и Шехтера заключается в том, что поле плотности имеет Гауссово распределение: 
\begin{equation}
P(\delta_M)d \delta_M = \frac{1}{ \sqrt{2} \pi \sigma _M} \exp(-\frac{\delta_{M}^{2}}{2\sigma_{M}^{2}})d \delta_M, 
\end{equation}
где $\sigma_M$ - дисперсия поля плотности, фильтрованного на масштабе R, заключаещого в себе массу M. Пусть $\delta_c$ - это некоторая критическая плотность, которая будет определена позже. Получаем, что вероятность того, что $\delta_M$ превышает $\delta_c$:
\begin{equation}
P_{> \delta_c} (M) = \int_{\delta_c}^{\infty} P(\delta_M)d \delta_M.
\end{equation}
Вероятность $(6)$ зависит от фильтр-массы M и от красного смещения z, так как дисперсия зависит от красного смещения z:
\begin{equation}
\sigma_{M}^{2} = \sigma_{R}^{2} = \frac{1}{2\pi^2} \int_{0}^{\infty} dkk^2P_M(k,z)\tilde{W}_{R}^{2}(k),
\end{equation}
$P_M$(k,z) - массовый спектр мощности, а $\tilde{W}_{R}^{2}$(k) - Фурье преобразование функции "top hat".

Наконец, функция масс n(M) должна быть определена как
\begin{equation}
\begin{split}
n(M)dM = \frac{\rho_0}{M}[P_{> \delta_c} (M) - P_{> \delta_c} (M + dM)] \\ = - \frac{\rho_0}{M}\frac{dP_{> \delta_c}}{dM}dM \\ = - \frac{\rho_0}{M}\frac{dP_{> \delta_c}}{d \sigma_M} \frac{d\sigma_M}{dM}dM.
\end{split}
\end{equation}
К сожалению, это формула не может претендавать на абсолютную точность, как минимум, потому, что она не учитывает участки пространства с пониженной плотностью (x: $\delta(x)<0$). В связи с этим, Пресс и Шехтер сделали еще одно предположение: участки пониженной плотности в конечном итоге аккрецируют на участки повышенной плотности, поэтому перед (8) был добавлен множитель 2.\\
Объединяя (5)-(8) и последнее замечание, имеем 
\begin{equation}
n(M)dM = -\sqrt{\frac{2}{\pi}} \frac{d\sigma_M}{dM}\frac{\rho_0\delta_c}{M\sigma_{M}^{2}}\exp(-\frac{\delta_{c}^{2}}{2\sigma_{M}^{2}})dM
\end{equation}

\section{Вычисление функции масс}
Стандартной процедурой является определение функции масс (9) путем вычисления дисперсии (7), используя массовый спект мощности из линейной теории.\\
В космологии возможно вычислить спектр мощности теоретически. Начальный спектр (initial) $P_i(k)\propto k$ (в данном случае "начальный" относится к эпохе с квантовыми флуктуациями до эпохи инфляции). Впоследствии этот спектр видоизменяется за счет того, что отдельные моды сначала выходят из под горизонта в эпоху инфлции, затем опять входят в него на стадии "обычного" расширения. Также на него влияет наличие давления у вещества: оно приводит к размыванию возмущений на определенных масштабах. Все эти процессы характеризуются так называемой передаточной функцией T(k) (Mo и др., [3]). Итоговый спектр мощности записывается в виде P(k) = $P_i(k)\cdot T^2$(k). Массовый спектр мощности может быть измерен довольно точно на некоторых масштабах с использованием космического микроволнового фонового излучения (реликтового излучения). Вид спектра представлен на Рис.3.\\
\begin{figure}[H]
\centering
\includegraphics[width=0.7\textwidth]{spectrum.png}
\caption{Массовый спектр мощности}  
\end{figure}

Этот спектр аппроксимируется следующим образом (Bardeen и др., 1986, [4]): \\
\begin{equation}
P(k) = A(T(k))^2k,
\end{equation}
\begin{equation}
T(k) = \frac{ln(1+2.34q)}{2.34q}(1+3.89q+(16.1q)^2+(5.64q)^3+(6.71q)^4)^{-\frac{1}{4}},
\end{equation}
q = $\frac{k}{\Omega_mh^2}, A\approx4.735\cdot10^6.$ \\
k выражается в h/Мпк, $\Omega_m \approx0.27, h = \frac{H}{100km/s/Mpc}\approx0.73$ - безразмерный параметр Хаббла.\\
Фурье преобразование функции-окна - $ \tilde{W}_R(k)$:
\begin{equation}
\tilde{W}_R(k) = \frac{3sin(kR) - 3kRcos(kR)}{(kR)^3}
\end{equation}
Теперь необходимо вычислить дисперсию (7), используя (10)-(12).  
\subsection{Приближенное интегрирование}
Прежде всего заметим, что массовый спектр (10)-(11) имеет две ассимптотики. $P\propto k$, если k$<<$1, и $P\propto k^{-3}$, если k$>>$1. \\
Кроме того, $\tilde{W}_R(k)$ может быть аппроксимирована:\\ 1) Если kR$<<$1, sin(kR)$\approx$kR и cos(kR)$\approx 1-(kR)^2/2$. \\
$\sigma_{M}^{2} = \frac{1}{2\pi^2} \int_{0}^{\infty} dkk^2P(k)$\\
2) Если kR$>>$1, 3kRcos(kR)$>>$3sin(kR). \\ \  $\sigma_{M}^{2} = \frac{1}{2\pi^2} \int_{0}^{\infty} dkk^2P(k)9(kR)^2cos^2(kR)/(kR)^6 = $, используя $cos^2(kR) = (1+cos(2kR))/2$ и теорему Римана об интеграле от осциллирующей функции $ = \frac{1}{2\pi^2} \int_{0}^{\infty} dkk^2P(k)\frac{9}{2(kR)^4}$\\
\\
Используя обе аппроксимации,вычислена дисперсия (7) на двух масштабах: $R>1/k_0$ и $R<1/k_0$, где $k_0\approx \Omega_m h^2$ - типичное значение k, при котором P(k) меняет свое поведение.\\
\begin{figure}[H]
\centering
\includegraphics[width=0.7\textwidth]{sigma2_appr.png}
\caption{Дисперсия $\sigma_{M}^{2}$. Результат получен приближенным интегрированием. Логарифмический масштаб}  
\end{figure}
Результат является хороший оценкой дисперсии (7) и отражает общую зависимость данной функции от M, однако оцененная ошибка при подсчете дисперсии при k$>>1$ оказалась относительно большой. Поэтому было решено вычислить дисперсию (7) численно. 
\subsection{Численное интегрирование с использованием Python}
Численное вычисление было произведено с использованием Python и его библиотек - scipy и numpy. Результаты подсчета дисперсии (7) и функции масс (9) приведены  на рисунках 5 и 6. Ниже приведен программный код вычисления и визуализации этих функций с комментариями.

\lstset{language=Python}
\begin{lstlisting}
import numpy as np
import matplotlib.pyplot as plt
from numpy import pi as pi
import scipy.integrate as integrate

A = 4.375e6 #constants
Omega = 0.27
h = 0.73
M_S = 1.98e30
Mpc = 3.0856776e22
rho = 0.3*9.31e-27*Mpc**3/h**3
sgm_s=1.686
k_0 = Omega*h**2
g = 4*pi*rho/3

def Rad (M): #radius R of a sphere with the mass M
    return (M/g)**(1/3)     
    
def P(x): #mass power spectrum (10)
    k = np.exp(x)
    return A*k*(T(k))**2
            
def T(x): #transitional function (11)
    q = x/(Omega*h**2)
    return np.log(1+2.34*q)*(2.34*q)**(-1)*(1+3.89*q+(16.1*q)**2+
    (5.64*q)**3+(6.71*q)**4)**(-1/4)

def W(x,R): #Fourier transform of the filter-function   
    k = np.exp(x)
    return (3*np.sin(k*R)-3*k*R*np.cos(k*R))/(k*R)**3
    #return 1/(R)**4/k**3    

def al(t, M): #al and br are auxiliary functions for calculating the variance (7)
    return np.exp(t)*g**(-1/3)*M**(1/3)
    
def br(t, M):
    a = al(t, M)
    return 3*np.sin(a) - 3*a*np.cos(a)
    
def sgm2(M): #sigma^2 - the variance (7)
    R = Rad (M)   
    return ( integrate.quad(lambda t:
    np.exp(3*t)*P(t)*(W(t,R))**2/(2*pi**2), np.log(1e-2), np.log(1e5))[0] )  

def der_sgm2(M):  #derivative of sigma^2      
    return ( integrate.quad(lambda t:
    np.exp(3*t)*P(t)*2*br(t,M)*(br(t,M)-al(t,M)**2*np.sin(al(t,M)))/
    (al(t,M)**6*M*2*pi**2), np.log(1e-2), np.log(1e5))[0] )

def dersgm(M): #derivative of the variance 
    return der_sgm2(M)/(2*np.sqrt(sgm2(M)))

def n(M): #mass function
    return np.sqrt(2/pi)*rho*dersgm(M)*sgm_s*
    np.exp(-sgm_s**2/(2*sgm2(M)))/(M*sgm2(M))

#Visualization:    
M_0 = M_S*1e5; M_F = M_S*1e15
x = np.logspace(np.log10(M_0), np.log10(M_F), num = 200) 
y = [sgm2(x[i]) for i in range (len(x))]
y1 = [n(x[i]) for i in range (len(x))]
y2 = [y1[i]*x[i] for i in range (len(x))]
plt.title('Mass function')
plt.xlabel('M [Solar Mass]/h')
plt.ylabel('dN(>M)/dlogM')
plt.xscale('log')
plt.yscale('log')
plt.plot (x*h/M_S, y2)
plt.show()
\end{lstlisting}



\begin{figure}[H]
\centering
\includegraphics[width=0.7\textwidth]{sigma2_scipy.png}
\caption{Дисперсия $\sigma_{M}^{2}$. Результат численного вычисления в Python. Логарифмический масштаб}  
\end{figure}

\begin{figure}[H]
\centering
\includegraphics[width=0.7\textwidth]{Massf.png}
\caption{Функция масс n(M). Результат численного вычисления в Python. Логарифмический масштаб}  
\end{figure}

\section{Анализ космологической симуляции}
Следующим этапом работы было научиться анализировать космологические симуляции, вычислять по ним поле плотности и функцию масс и сравнивать полученные зависимости с предсказанными теоретически. Сергеем Владимировичем Пилипенко была предоставлена космологическая симуляция размера 32Мпк*32Мпк*32Мпк с 512$^3$ частицами. Большое количество частиц требует использовать нетривиальные алгоритмы подстчета поля потности. Общая схема алгоритма такова: пространство разделено на 100$^3$ ячеек и покрыто k$\cdot$100$^3$ сферами радиуса R. k - положительное рациональное число, выбираемое в соответствии с требуемой точностью. Далее подсчитывается количество частиц, попадающих в каждую сферу, полученное число умножается на массу одной частицы и делится на объем сферы. Кроме того, часть программного кода (особенно обращения к длинным массивам) были переписаны на C и вставлены в Python, используя библиотеку scipy.weave, что дало увеличение скорости работы программы в 27 раз. Однако общей программной средой по-прежнему остается среда Python, так как Python имеет незаменимые библотеки pygadgetreader (для чтения космологических симуляций), numpy и другие. Ниже приведен программный код вычисления поля плотности с комментариями.
\\
Результат вычисления представлен на рисунке 7.
\lstset{language=Python}
\begin{lstlisting}
from __future__ import division #so that 3/2 = 1.5, 
#but not 1 as in Python2.* 
import pygadgetreader as pgr
import numpy as np
from numpy import pi
from time import time
from scipy import weave

t0 = time()
coord = pgr.readsnap ('rsnap_002.dat', 'pos', 1) 
t1 = time() - t0
maxcoord = 32
numofcells = 100
hx, hy, hz = maxcoord/numofcells, maxcoord/numofcells, maxcoord/numofcells
numofsph = numofcells + 1 # = number of knots
R = 0.999*hx
V = 4/3*pi*R**3 

nop = int(np.size(coord)/3) #number of particles
rho = 1/V
xmax, ymax, zmax = 32, 32, 32
xmin, ymin, zmin = 0, 0, 0
mx = int((xmax - xmin)/hx)
my = int((ymax - ymin)/hy)
mz = int((zmax - zmin)/hz)
part = np.zeros((3))
density = np.zeros((numofsph,numofsph,numofsph))
r1 = int(R/hx) + 1; r2 = int(R/hy) + 1; r3 = int(R/hz) + 1

def boundcond ():
    density[100,:,:] = density [0, :, :]
    density[:,100,:] = density [:, 0, :]
    density[:,:,100] = density [:, :, 0]

ccode = \
        """
        part[0] = coord[i*3]; part[1] = coord[i*3+1];
        part[2] = coord[i*3+2];
        """
ccode2 = \
        """
        int index = 0;
        int meff, neff, peff;
        meff = m + i; neff = n + j; peff = p + k;
        if (meff<0) {meff+=numofsph-1;}
        if (neff<0) {neff+=numofsph-1;}
        if (peff<0) {peff+=numofsph-1;}
        if (meff>=numofsph) {meff-=numofsph+1;}
        if (neff>=numofsph) {neff-=numofsph+1;}
        if (peff>=numofsph) {peff-=numofsph+1;}
        index = peff+neff*numofsph+meff*numofsph*numofsph;
        density[index]+=rho;
        """
for i in range (nop):
    weave.inline (ccode, ['i', 'part', 'coord'], compiler = 'gcc')
    #part = coord[i]
    x = part[0]; y = part[1]; z = part[2]    
    m = int(x/hx); n = int(y/hy); p = int(z/hz)
    for i in range (-r1+1, r1+1):
        for j in range (-r2+1, r2+1):
            for k in range (-r3+1, r3+1):
                xsp = (m+i)*hx; ysp = (n+j)*hy; zsp = (p+k)*hz
                if ((x-xsp)**2 + (y-ysp)**2 + (z - zsp)**2 <= R**2):
                    weave.inline (ccode2, ['density', 'm', 'i', 'n', 'j', 'p', 'k', 
                    'rho', 'numofsph'], compiler = 'gcc')                    
boundcond()
     
\end{lstlisting}
\begin{figure}[H]
\centering
\includegraphics[width=0.7\textwidth]{sim.jpg}
\caption{X, Y и Z "срезы" вычисленного поля плотности}  
\end{figure}

\section{Выводы и планы на следующий семестр}
В ходе работы были вычислены дисперсия (7) и функция масс (9) приближенным интегрированием и численно в рамках формализма Пресса-Шехтера, [2]. Кроме того, была обработана космологическая симуляция с большим количеством частиц (512$^3$) и построено соответсвующее поле плотности. Наконец, произведена оптимизация работы программы за счет использования комбинации языков программирования Python и C.\\
План работ на следующий семестр включает в себя продолжение анализа данной космологической симуляции и проверку модифицированной теории Пресса-Шехтера на малых массах. Первостепенная задача:\\
•	Разбить сферы по контрастам плотности на небольшое количество групп;\\
•	Используя библиотеку RockStar Halo Finder, определить распределение гало темной материи по данному кубу и распределить гало по сферам;\\
•	Построить функцию масс гало темной материи для данной симуляции и сравнить с функцией масс, предсказанной теоретически.\\
•	Учесть фактор роста, чтобы получить достоверную информацию при z = 0, так как симуляция предоставлена при z = 1 (Информация о факторе роста - Barkana, Loeb. The first sources of light and the reionization of the universe (PR349, 2001), [4]).

\begin{thebibliography}{9}
\bibitem{Misner}
Misner C.W., Thorne K.S., Wheeler J.A., Gravitation, 1973, W.H. Freeman, 1279 pp.
\bibitem{PrssSch}
Press W.H., Schechter P., 1974. Astrophys. J. 187, 425.
\bibitem{Bosch}
Mo H., Bosch F., White S., 2010, Galaxy formation and evolution,  Cambridge University Press, 840 pp.
\bibitem{Barkana}
Barkana R., Loeb A., 2001, Physics Reports 349, 125-238.
\bibitem{Arhipova}
Arkhipova N.A., Komberg B.V., Lukash V.N., Miheeva E.V., 2007, Astronomy J., 84(10), 874-884
\end{thebibliography}
\end{document}